%!TEX program=xelatex
% \title{QUXIAOFENG-CV-CN}
%%%%%%%%%%%%%%%%%%%%%%%%%%%%%%%%%%%%%%%
% This is a modified ONE COLUMN version of
% the following template:
% 
% Deedy - One Page Two Column Resume
% LaTeX Template
% Version 1.1 (30/4/2014)
%
% Original author:
% Debarghya Das (http://debarghyadas.com) 
%
% Original repository:
% https://github.com/deedydas/Deedy-Resume
%
% IMPORTANT: THIS TEMPLATE NEEDS TO BE COMPILED WITH XeLaTeX
%
% This template uses several fonts not included with Windows/Linux by
% default. If you get compilation errors saying a font is missing, find the line
% on which the font is used and either change it to a font included with your
% operating system or comment the line out to use the default font.
% 
%%%%%%%%%%%%%%%%%%%%%%%%%%%%%%%%%%%%%%
% 
% TODO:
% 1. Integrate biber/bibtex for article citation under publications.
% 2. Figure out a smoother way for the document to flow onto the next page.
% 3. Add styling information for a "Projects/Hacks" section.
% 4. Add location/address information
% 5. Merge OpenFont and MacFonts as a single sty with options.
% 
%%%%%%%%%%%%%%%%%%%%%%%%%%%%%%%%%%%%%%
%
% CHANGELOG:
% v1.1:
% 1. Fixed several compilation bugs with \renewcommand
% 2. Got Open-source fonts (Windows/Linux support)
% 3. Added Last Updated
% 4. Move Title styling into .sty
% 5. Commented .sty file.
%
%%%%%%%%%%%%%%%%%%%%%%%%%%%%%%%%%%%%%%%
%
% Known Issues:
% 1. Overflows onto second page if any column's contents are more than the
% vertical limit
% 2. Hacky space on the first bullet point on the second column.
%
%%%%%%%%%%%%%%%%%%%%%%%%%%%%%%%%%%%%%%

\documentclass[]{deedy-resume-openfont}

\usepackage{xeCJK}
\setCJKmainfont[
  BoldFont={WenQuanYi Micro Hei},
  ItalicFont={AR PL UKai CN}]
{AR PL UMing CN}
\setCJKsansfont{WenQuanYi Micro Hei}
\usepackage[math-style=TeX,vargreek-shape=unicode]{unicode-math}
\usepackage{bibentry}
\nobibliography*

\begin{document}

%%%%%%%%%%%%%%%%%%%%%%%%%%%%%%%%%%%%%%
%
%     LAST UPDATED DATE
%
%%%%%%%%%%%%%%%%%%%%%%%%%%%%%%%%%%%%%%
\lastupdated

%%%%%%%%%%%%%%%%%%%%%%%%%%%%%%%%%%%%%%
%
%     TITLE NAME
%
%%%%%%%%%%%%%%%%%%%%%%%%%%%%%%%%%%%%%%

\namesection{曲晓峰}{}{ \urlstyle{same}
\href{mailto:quxiaofeng@ieee.org}{quxiaofeng@ieee.org} | +86 181 1879 1848}

% \href{http://github.com/quxiaofeng}{github.com/quxiaofeng}

% \begin{center}
% \huge\color{subheadings}\custombold{软硬结合\textbullet{}敏捷实效\textbullet{}精简优雅}
% \end{center}

%%%%%%%%%%%%%%%%%%%%%%%%%%%%%%%%%%%%%%
%     Experience
%%%%%%%%%%%%%%%%%%%%%%%%%%%%%%%%%%%%%%

\section{工作经历}

\runsubsection{清华大学深圳研究生院}
\descript{| 博士后 }
\location{2018 年 1 月 – 至今 | 深圳}
\begin{tightemize}
\item 研究方向:机器视觉、生物特征识别、图像采集与识别系统、视觉物体检测与测量、自动光学缺陷检测
\item 绿米联创合作项目(2017 年 9 月 – 2019 年 10 月,小米生态链企业,主营低功耗 Zigbee 物联网智能家居智能硬件)
\begin{itemize}
\item Zigbee 网关 LED 生产质量在线检测项目:使用 USB Dongle 通过 Zigbee 发送 LED 调试指令,检测 LED 是否正常工作。
\item 智能门锁安装项目:以小程序为入口,利用深度学习(物体检测与实例分割),在手机拍摄并上传的门锁图像中,识别门锁类型,检测门锁导向片的组件,并测量导向片的尺寸,生成相关数据供上门安装人员预先准备安装辅助工具。该项目将上门安装时间从14天缩短到4天,支撑了七种、三万把以上的智能门锁的上门安装。
\item 智能门锁算法优化项目:通过纹理分析,检测指纹传感器上的非指纹异物;分析三维人脸传感器采集到的点云数据质量,分析三维人脸识别算法的可靠性。
\item 新型智能家居产品调研;智能家居、智能硬件产品相关知识产权布局。
\end{itemize}
\item 门把手识别:纹理特征分析;进一步小型化的笔式识别系统原型。
\item 可撤销的手上生物特征识别系统原型开发及专利申请。
\end{tightemize}
\sectionsep

% \runsubsection{Lumi United Service}
% \descript{| 算法工程师 }
% \location{2017 年 9 月 – 至今 | 香港}
% %\vspace{\topsep} % Hacky fix for awkward extra vertical space
% \begin{tightemize}
% \item 嵌入式产品算法、深度学习应用、图像与视频算法
% \end{tightemize}
% \sectionsep

\runsubsection{机器视觉创业项目}
\descript{| }
\location{2017 年 5 月 – 2017 年 9 月 | 深圳}
\begin{tightemize}
\item 机器视觉项目开发:客户对接、需求分析、项目规划、系统搭建、进场测试、项目移交。
\item 建立机器视觉团队:人员规划、招聘面试、入职培训、团队建设。
\item 电子纸14项表面缺陷检测项目、比亚迪塑料手机壳表面缺陷检测项目、比亚迪3D玻璃手机壳表面质量和三维检测项目。
\end{tightemize}
\sectionsep

\runsubsection{绿米联创}
\descript{| 首席算法工程师(实习) }
\location{2016 年 9 月 – 2017 年 4 月 | 深圳}
\begin{tightemize}
\item 带有 Zigbee 网关功能的 WIFI 大视角摄像头项目:设计成像质量检测方法,并与供应商协调优化视频参数提高成像质量;通过声音通讯传递联网信息,进行快速联网设置。
\item 嵌入式产品识别算法开发;数据分析/深度学习内部培训;推动并组建内部数据分析团队与流程。
\end{tightemize}
\sectionsep

\runsubsection{香港理工大学纺织及制衣学系}
\descript{| 助理研究员(兼职) }
\location{2013 年 6 月 – 2014 年 2 月 | 香港}
\begin{tightemize}
\item 利用 Python 和 OpenGL 开发交互式 3D 人体模型
\end{tightemize}
\sectionsep

\runsubsection{\href{http://www.comp.polyu.edu.hk/}{香港理工大学电子计算学系}}
\descript{| 助教(兼职) }
\location{2010 年秋季学期 - 2012 年秋季学期 | 香港}
\begin{tightemize}
\item COMP407 Computer Graphics,计算机图形学;COMP102 Enterprise Information Technology,企业信息技术;COMP210 Discrete Structures, 离散数据结构;COMP435p Biometrics Authentication, 人体生物特征认证;ENG2003 Information Technology, 信息技术。
\end{tightemize}
\sectionsep

\runsubsection{国际图形图像杂志 (International Journal of Image and Graphics)}
\descript{| 主编助理(兼职) }
\location{2010 年 9 月 - 2012 年 3 月 | 香港}
\begin{tightemize}
\item 管理组织五期学术杂志的收稿、审稿与发表相关工作。
\end{tightemize}
\sectionsep

\runsubsection{人体生物特征识别研究中心}
\descript{| 助理研究员 }
\location{2008 年 4 月 - 2010 年 3 月 | 香港理工大学电子计算学系}
\begin{tightemize}
\item 基于 51 单片机/Cypress USB/三星 ARM/德州仪器 DSP/Altera FPGA 开发了一系列生物特征识别系统。
\end{tightemize}
\sectionsep

\sectionsep

%%%%%%%%%%%%%%%%%%%%%%%%%%%%%%%%%%%%%%
%     EDUCATION
%%%%%%%%%%%%%%%%%%%%%%%%%%%%%%%%%%%%%%

\section{教育背景}

\runsubsection{\href{www.polyu.edu.hk}{香港理工大学}}
\descript{| 2017 年 3 月}
\location{哲学博士 | 计算机专业 | 中国 \textbullet{} 香港}

\href{www.polyu.edu.hk}{香港理工大学} \textbullet{} \href{http://www.comp.polyu.edu.hk/}{电子计算学系} \textbullet{} \href{http://www.comp.polyu.edu.hk/~biometrics/}{人体生物特征识别中心}
\sectionsep

\runsubsection{\href{http://www.sut.edu.cn/}{沈阳工业大学}}
\descript{| 2008 年 3 月}
\location{工学硕士 | 检测技术及自动化装置专业 | 视觉检测技术研究所 | 中国\textbullet{}沈阳}

第十届全国大学生课外学术科技作品竞赛挑战杯三等奖 \textbullet{} 优秀毕业论文 \textbullet{} 优秀毕业生 
\sectionsep

\runsubsection{\href{http://www.sut.edu.cn/}{沈阳工业大学}}
\descript{| 2005 年 7 月}
\location{工学学士 | 电子信息工程专业 | 辅修工商管理专业合格 | 中国\textbullet{}沈阳}

三好学生 \textbullet{} 优秀团员 \textbullet{} 连续三年奖学金\footnote{毕业学年无奖学金评定} \textbullet{} 优秀毕业设计 \textbullet{} 优秀毕业生


% 填空
\vfill

% 翻页
\newpage


%%%%%%%%%%%%%%%%%%%%%%%%%%%%%%%%%%%%%%
%     LANGUAGES AND SKILLS
%%%%%%%%%%%%%%%%%%%%%%%%%%%%%%%%%%%%%%

\section{语言}
\begin{minipage}[t]{.6\textwidth}
\subsection{程序语言}
\location{有开发经验}
Python \textbullet{}
C \textbullet{}
C++ \textbullet{}
MATLAB \textbullet{}
{\color{red} $\varheartsuit$} Julia \textbullet{}
\LaTeX \\
\location{熟悉}
Ruby \textbullet{}
Javascript \textbullet{}
HTML \textbullet{}
CSS \textbullet{}
Bash \textbullet{}
Android \textbullet{}
MySQL
\sectionsep
\end{minipage}
\hfill
\begin{minipage}[t]{.35\textwidth}
\subsection{自然语言}
\location{专业协作} 汉语,英语\\
\location{文献阅读} 法语、俄语、德语\\
\end{minipage}

\sectionsep

%%%%%%%%%%%%%%%%%%%%%%%%%%%%%%%%%%%%%%
%     ASSOCIATIONS
%%%%%%%%%%%%%%%%%%%%%%%%%%%%%%%%%%%%%%

\section{专业协会}

\runsubsection{IEEE 香港分会 / 中国计算机协会 CCF / 香港互联网协会 ISOCHK}
\descript{| 会员}
\sectionsep
\runsubsection{雷锋网}
\descript{| 专栏作者}

\sectionsep


% %%%%%%%%%%%%%%%%%%%%%%%%%%%%%%%%%%%%%%
% %     PUBLICATION
% %%%%%%%%%%%%%%%%%%%%%%%%%%%%%%%%%%%%%%

\renewcommand\refname{论文} %changes default name to Publications
\nocite{*} %lists everything in the .bib file
\bibliographystyle{apalike} %hundreds of styles to choose from
\bibliography{pub} %name of your .bib file

\sectionsep

%\begin{tightemize}
%\item \bibentry{Qu2015lps}
%\item \bibentry{Xie2012}
%\item \bibentry{Qu2008pca}
%\end{tightemize}
%\sectionsep

%%%%%%%%%%%%%%%%%%%%%%%%%%%%%%%%%%%%%%
%     PATENTS
%%%%%%%%%%%%%%%%%%%%%%%%%%%%%%%%%%%%%%

\section{发明专利}


\runsubsection{门锁安装辅助方法及装置}
\descript{| 中国专利 ZL201810044199 }
\location{2020 年二月十八日授权}
\sectionsep

\runsubsection{ 图像识别方法、装置、系统、电子设备及存储介质 }
\descript{| 中国专利 ZL201910855388 }
\location{2019 年十二月十三日公开}
\sectionsep

\runsubsection{ 指纹识别方法、装置、智能门锁及可读存储介质 }
\descript{| 中国专利 ZL201910663347 }
\location{2019 年十一月二十九日公开}
\sectionsep

\runsubsection{ 磁悬浮灯及其控制方法 }
\descript{| 中国专利 ZL201910527245 }
\location{2019 年十一月五日公开}
\sectionsep

\runsubsection{ 电磁悬浮开合的磁吸附式吊灯 }
\descript{| 中国专利 ZL201910452716 }
\location{2019 年十月二十二日公开}
\sectionsep

\runsubsection{握姿人手图像识别系统及其识别方法}
\descript{| 中国专利 CN201410137490 }
\location{2019 年十月十五日授权}
\sectionsep

\runsubsection{ 导向片参数获取方法、装置、电子设备以及存储介质 }
\descript{| 中国专利 ZL201910324817 }
\location{2019 年九月六日公开}
\sectionsep

\runsubsection{ 生物特征模型生成方法、装置、服务器及存储介质 }
\descript{| 中国专利 ZL201910309876 }
\location{2019 年八月二十七日公开}
\sectionsep

\runsubsection{ 设备控制方法、装置、摄像机以及存储介质 }
\descript{| 中国专利 ZL201910228907 }
\location{2019 年八月九日公开}
\sectionsep

\runsubsection{ 设备联动控制方法、装置、系统、网关及存储介质 }
\descript{| 中国专利 ZL201910204894 }
\location{2019 年七月三十日公开}
\sectionsep

\runsubsection{ 一种掌纹识别方法及系统 }
\descript{| 中国专利 ZL201910229458 }
\location{2019 年七月十二日公开}
\sectionsep

\runsubsection{ 一种微型人手识别设备及识别方法 }
\descript{| 中国专利 ZL201910224288 }
\location{2019 年七月十二日公开}
\sectionsep

\runsubsection{ 基于物体识别的信息推送方法、装置、电子设备及系统 }
\descript{| 中国专利 ZL201910118124 }
\location{2019 年七月二日公开}
\sectionsep

\runsubsection{ 数据处理方法、装置及系统 }
\descript{| 中国专利 ZL201910065331 }
\location{2019 年六月二十八日公开}
\sectionsep

\runsubsection{ 跌倒预测方法、装置、电子设备及系统 }
\descript{| 中国专利 ZL201910100602 }
\location{2019 年六月二十一日公开}
\sectionsep

\runsubsection{ 模型配置方法、装置、电子设备及可读取存储介质 }
\descript{| 中国专利 ZL201910105813 }
\location{2019 年六月十四日公开}
\sectionsep

\runsubsection{ 智能家居设备推荐方法、装置、物联网系统以及云服务器 }
\descript{| 中国专利 ZL201910065330 }
\location{2019 年六月七日公开}
\sectionsep

\runsubsection{ 视觉传感器及应用于视觉传感器的物体检测方法和装置 }
\descript{| 中国专利 ZL201910024672 }
\location{2019 年五月二十四日公开}
\sectionsep

\runsubsection{ 配网的方法、装置、智能家居系统、设备以及存储介质 }
\descript{| 中国专利 ZL201811528413 }
\location{2019 年五月十七日公开}
\sectionsep

\runsubsection{ 设备状态显示方法、装置、终端及存储介质 }
\descript{| 中国专利 ZL201811500376 }
\location{2019 年五月十七日公开}
\sectionsep

\runsubsection{ 设备控制方法、装置、系统及存储介质 }
\descript{| 中国专利 ZL201811528405 }
\location{2019 年四月五日公开}
\sectionsep

\runsubsection{ 一种隐藏式屏下光学模组及电子设备 }
\descript{| 中国专利 ZL201811474036 }
\location{2019 年四月五日公开}
\sectionsep

\runsubsection{ 目标识别方法、装置、视觉传感器及智能家居系统 }
\descript{| 中国专利 ZL201811313873 }
\location{2019 年三月二十九日公开}
\sectionsep

\runsubsection{ 智能家居设备控制方法、装置、系统及存储介质 }
\descript{| 中国专利 ZL201811528407 }
\location{2019 年三月十九日公开}
\sectionsep

\runsubsection{ 实现智能设备推荐的方法及装置 }
\descript{| 中国专利 ZL201811075500 }
\location{2019 年三月一日公开}
\sectionsep

\runsubsection{ 设备网络配置方法、装置及服务器 }
\descript{| 中国专利 ZL201811475972 }
\location{2019 年二月二十六日公开}
\sectionsep

\runsubsection{ 论文发表方法、装置以及服务器 }
\descript{| 中国专利 ZL201811163103 }
\location{2019 年二月二十二日公开}
\sectionsep

\runsubsection{ 在线竞赛方法、装置及服务器 }
\descript{| 中国专利 ZL201811163099 }
\location{2019 年二月十五日公开}
\sectionsep

\runsubsection{一种基于移动端非接触式指掌识别方法及系统}
\descript{| 中国专利 ZL201810903071 }
\location{2019 年一月四日公开}
\sectionsep

\runsubsection{手语控制方法、装置及系统}
\descript{| 中国专利 ZL201810846966 }
\location{2018 年十二月十八日公开}
\sectionsep

\runsubsection{门锁安装辅助方法及装置}
\descript{| 中国专利 ZL201810043800 }
\location{2018 年六月一日公开}
\sectionsep

\runsubsection{一种人体掌纹图像采集装置及处理方法}
\descript{| 中国专利 ZL20110362063 }
\location{2014 年二月二十一日授权}
\sectionsep

\runsubsection{圆筒内外壁加工精度在线成像检测装置及在线成像检测方法}
\descript{| 中国专利 ZL200610155870 }
\location{2009 年六月十日授权}
\sectionsep


% 填空
\vfill

% 翻页
\newpage

%%%%%%%%%%%%%%%%%%%%%%%%%%%%%%%%%%%%%%
%     PROJECTS
%%%%%%%%%%%%%%%%%%%%%%%%%%%%%%%%%%%%%%

\section{项目与系统}

\runsubsection{门把手人手识别系统}
\descript{} %\descript{| 孙伟}
\location{门把手外形的人体工学的人手生物特征识别系统 \\
\urlstyle{same} \url{http://www.quxiaofeng.me/doorknob}}
\sectionsep
\runsubsection{线扫描掌纹识别系统}
\descript{} %\descript{| 张栓伟、周剑}
\location{基于 CIS 线扫描图像传感器的掌纹采集与识别系统 \\
\urlstyle{same} \url{http://www.quxiaofeng.me/lps}}
\sectionsep
\runsubsection{手背纹理采集识别系统}
\descript{} %\descript{| 谢晋 }
\sectionsep
\runsubsection{基于 USB 数字相机的高精度指纹采集系统}
\descript{} %\descript{| 刘凤 }
\sectionsep
\runsubsection{使用 CY7C68013A USB 接口的多通道气味传感器}
\descript{} %\descript{| 郭冬敏 }
\sectionsep
\runsubsection{使用 TI DSP、Altera FPGA 和 三星 ARM 的基于结构光的 3D 掌纹识别系统}
\descript{} %\descript{| 骆南、黎伟 }
\sectionsep
\runsubsection{基于 TI DM642 DSP 的嵌入式掌纹识别系统}
\descript{} %\descript{| 骆南、郭振华}
\sectionsep

\sectionsep

\end{document}  \documentclass[]{article}